\documentclass[12pt]{article}

\usepackage{packages}
\usepackage{environments}
\usepackage{commands}
\usepackage{titlepage}

\DeclareMathOperator{\omult}{\odot}

\usepackage{tikz}
\usepackage{mathdots}
\usepackage{yhmath}
\usepackage{cancel}
\usepackage{color}
\usepackage{siunitx}
\usepackage{array}
\usepackage{multirow}
\usepackage{amssymb}
\usepackage{tabularx}
\usepackage{booktabs}
\usepackage{algorithm}
\usepackage{algpseudocode}
\usetikzlibrary{fadings}
\usetikzlibrary{patterns}
\usetikzlibrary{shapes}

\begin{document}

    \section*{Алгоритм}

    \begin{algorithm}
        \caption{Sample Caption}
        \begin{algorithmic}
            \Require $n \geqslant 0$
            \Ensure $y = x^{n}$
            \State $y \gets 1$
        \end{algorithmic}
    \end{algorithm}

    \section*{Доказательство корректности}

    \section*{Оценка сложности}

    \section*{Оценка потребляемой памяти}

\end{document}